% datum?
\documentclass[thesis=B,czech]{FITthesis}[2019/03/06]

\usepackage[utf8]{inputenc} % LaTeX source encoded as UTF-8

\usepackage{todonotes}

\department{Katedra softwarového inženýrství}
\title{Nástroj pro konfiguraci a monitorování}
\authorGN{Václav}
\authorFN{Kubernát}
\authorWithDegrees{Václav Kubernát}
\author{Václav Kubernát}
\supervisor{Ing. Tomáš Čejka, Ph.D.}
\acknowledgements{Doplňte, máte-li komu a za co děkovat. V~opačném případě úplně odstraňte tento příkaz.}
% TODO: dopsat
\abstractCS{V~několika větách shrňte obsah a přínos této práce v~češtině. Po přečtení abstraktu by se čtenář měl mít čtenář dost informací pro rozhodnutí, zda chce Vaši práci číst.}
% TODO: dopsat
\abstractEN{Sem doplňte ekvivalent abstraktu Vaší práce v~angličtině.}
\placeForDeclarationOfAuthenticity{V~Praze}
% TODO: zkontrolovat
\declarationOfAuthenticityOption{4} %volba Prohlášení (číslo 1-6)
% TODO: dopsat
\keywordsCS{Nahraďte seznamem klíčových slov v češtině oddělených čárkou.}
% TODO: dopsat
\keywordsEN{Nahraďte seznamem klíčových slov v angličtině oddělených čárkou.}
% TODO: mám sem hodit link na git?
% \website{http://site.example/thesis} %volitelná URL práce, objeví se v tiráži - úplně odstraňte, nemáte-li URL práce

\begin{document}

% \newacronym{CVUT}{{\v C}VUT}{{\v C}esk{\' e} vysok{\' e} u{\v c}en{\' i} technick{\' e} v Praze}
% \newacronym{FIT}{FIT}{Fakulta informa{\v c}n{\' i}ch technologi{\' i}}

\begin{introduction}
TODO: krátce napsat, k čemu ten program je

TODO: zmínit NETCONF

TODO: napsat, komu se to hodí (cesnet síťaři) a (krátce) co měli doteď

TODO: \ldots a možná že se to bude používat na nějakých zařízeních (zjistit od Honzy, jak to přesně je)
\end{introduction}


\chapter{Cíl práce} \todo{Jsou cíle nutné? V podstatě tu jen opisuju zadání\ldots Ideálně bych to mergenul do úvodu nebo tak}
Cílem práce je vytvořit konzolovou aplikaci, pomocí které lze vzdáleně konfigurovat zařízení pomocí protokolu NETCONF\@. Aplikace bude intuitivní, uživatel by ji měl být schopný ovládat i bez detailní znalosti NETCONF\@. Příkazy a jejich argumenty bude možné stiskem klávesy automaticky doplňovat.


\chapter{Existující relevantní práce}
TODO: Popsat detailněji, co je třeba, aby ten program dělal/uměl (nějaký use-case nebo tak něco) a že to současné programy neumí
\section{Netopeer2-cli}
TODO: Dopsat
\section{sysrepocfg}
TODO: Konkrétně popsat výhody, nevýhody (například, že to nemá nic společného s netconfem xd)
\section{TODO: Mrknout na další programy (ne nutně cesneťácké)}


\chapter{Návrh}
TODO: něco vymyslet sem možná (nebo možná ne)

\section{Protokoly a použité technologie?}
TODO: dopsat o NETCONFU a YANGu, zmínit sysrepo

\section{Koncepce programu}
TODO: koncepce programu: interaktivní konzolová aplikace (shell) -> vysvětlit
TODO: přidat příklad použití (nějakej copy paste z shellu)

\section{Datový model}
TODO: Popsat syntaxi a AST


\chapter{Implementace}
TODO: dopsat něco o použitém jazyku a proč zrovna C++

\section{Součásti programu}
TODO: obrázek kudy jdou data; pracovní cyklus programu (nějaký vývoják input -> parser -> interpreter\ldots)

\subsection{Uživatelské rozhraní}
TODO: napsat, že používám "readline" atd.
TODO: ukázat replxx a proč nepoužívám readline/linenoise
TODO: ukázat docopt.cpp

\subsection{Parser}
největší část programu
TODO: ukázat Boost Spirit
TODO: ukázat gramatiky
TODO: ukázat libyang

\subsection{Interpreter}
TODO: dopsat (vezme user input a zavolá metody/funkce\ldots)

\subsection{DatastoreAccess}
TODO: dopsat (abstraktní rozhraní pro přístup do datastoru\ldots)

\subsubsection{NetconfAccess}
TODO: ukázat libnetconf2

\subsubsection{SysrepoAccess}
TODO: ukázat sysrepo klientskou knihovnu

\subsection{Testování}
TODO: ukázat doctest
TODO: ukázat trompeloeil
TODO: zmínit Catch

\subsection{Knihovna na logování?}
TODO: spdlog, použito jen v netconf cpp wrapperu\ldots od Honzy


\chapter{Vyhodnocení}
\section{Testování}
TODO: testování přes doctest, především unit testy, ale i nějaký jiný (sysrepo.cpp)\@. hlášení bugů přes issue tracker, po opravení napsán test
\section{Porovnání s konkurencí}
TODO: popsat, že ačkoliv moje aplikace nepodporuje 100 \% YANGu/NETCONFu, na základní použití stačí (hlavní cíl byl, upravovat konfiguraci bez znalosti YANG modelů a NETCONFu)
\section{Výsledky z nasazení}
TODO: no, prej to Honza používá :D

\begin{conclusion}
TODO: Něco dopsat sem, asi v závislosti na zbytku
\end{conclusion}

\bibliographystyle{csn690}
\bibliography{mybibliographyfile}

\appendix

\chapter{Seznam použitých zkratek}
% \printglossaries
\begin{description}
	\item[GUI] Graphical user interface
	\item[XML] Extensible markup language
\end{description}

\chapter{Obsah přiloženého CD}

\begin{figure}
	\dirtree{
	}
\end{figure}

\end{document}

% datum?
\documentclass[thesis=B,czech]{FITthesis}[2019/03/06]

\usepackage[utf8]{inputenc} % LaTeX source encoded as UTF-8

% my packages and stuff
\usepackage{todonotes}
\usepackage{xevlna}
% from https://tex.stackexchange.com/a/251025
\usepackage{relsize}
\usepackage{xspace}
\newcommand{\Rplus}{\protect\hspace{-.1em}\protect\raisebox{.35ex}{\smaller{\smaller\textbf{+}}}}
\newcommand{\Cpp}{\mbox{C\Rplus\Rplus}\xspace}


\department{Katedra softwarového inženýrství}
\title{Nástroj pro konfiguraci a monitorování}
\authorGN{Václav}
\authorFN{Kubernát}
\authorWithDegrees{Václav Kubernát}
\author{Václav Kubernát}
\supervisor{Ing. Tomáš Čejka, Ph.D.}
% TODO: dopsat
\acknowledgements{Doplňte, máte-li komu a za co děkovat. V~opačném případě úplně odstraňte tento příkaz.}
% TODO: dopsat
\abstractCS{V~několika větách shrňte obsah a přínos této práce v~češtině. Po přečtení abstraktu by se čtenář měl mít čtenář dost informací pro rozhodnutí, zda chce Vaši práci číst.}
% TODO: dopsat
\abstractEN{Sem doplňte ekvivalent abstraktu Vaší práce v~angličtině.}
\placeForDeclarationOfAuthenticity{V~Praze}
% TODO: zkontrolovat
\declarationOfAuthenticityOption{4} %volba Prohlášení (číslo 1-6)
% TODO: dopsat
\keywordsCS{Nahraďte seznamem klíčových slov v češtině oddělených čárkou.}
% TODO: dopsat
\keywordsEN{Nahraďte seznamem klíčových slov v angličtině oddělených čárkou.}
% TODO: mám sem hodit link na git?
% \website{http://site.example/thesis} %volitelná URL práce, objeví se v tiráži - úplně odstraňte, nemáte-li URL práce

\begin{document}

\begin{introduction}
% TODO: krátce napsat, k čemu ten program je

% TODO: zmínit NETCONF

% TODO: napsat, komu se to hodí (cesnet síťaři) a (krátce) co měli doteď

% TODO: ... a možná že se to bude používat na nějakých zařízeních (zjistit od Honzy, jak to přesně je)
\end{introduction}


\chapter{Cíl práce} \todo{Jsou cíle nutné? V podstatě tu jen opisuju zadání\ldots Ideálně bych to mergenul do úvodu nebo tak}
Cílem práce je vytvořit konzolovou aplikaci, pomocí které lze vzdáleně konfigurovat zařízení pomocí protokolu NETCONF\@. Aplikace bude intuitivní, uživatel by ji měl být schopný ovládat i bez detailní znalosti NETCONF\@. Příkazy a jejich argumenty bude možné stiskem klávesy automaticky doplňovat.


\chapter{Existující relevantní práce}
% TODO: Popsat detailněji, co je třeba, aby ten program dělal/uměl (nějaký use-case nebo tak něco) a že to současné programy neumí
\section{Netopeer2-cli}
% TODO: Dopsat
\section{sysrepocfg}
% TODO: Konkrétně popsat výhody, nevýhody (například, že to nemá nic společného s netconfem xd)
\section{TODO Mrknout na další programy (ne nutně cesneťácké)}


\chapter{Návrh}
% TODO: něco vymyslet sem možná (nebo možná ne)

\section{Protokoly a použité technologie?}
% TODO: dopsat o NETCONFU a YANGu, zmínit sysrepo

\section{Koncepce programu}
% TODO: koncepce programu: interaktivní konzolová aplikace (shell) -> vysvětlit
% TODO: přidat příklad použití (nějakej copy paste z shellu)

\section{Datový model}
% TODO: Popsat syntaxi a AST


\chapter{Implementace}
% TODO: dopsat něco o použitém jazyku a proč zrovna \Cpp{}

\section{Součásti programu}
% TODO: obrázek kudy jdou data; pracovní cyklus programu (nějaký vývoják input -> parser -> interpreter...)

\subsection{Uživatelské rozhraní}
Uživatelské rozhraní je poměrně přímočaré: program cyklicky načítá řádky uživatelského vstupu a zpracovává je. K interaktivnímu používání aplikace, je nutné implementovat nějaký způsob editace příkazové řádky. To zahrnuje především:
\begin{itemize}
\item možnost úpravy aktuálního obsahu příkazové řádky před odesláním
\item klávesové zkratkty \todo{Reword this}
\item automatické doplňování příkazů pomocí klávesové zkratky
\item ukládání historie příkazů
\end{itemize}
Ideální knihovna by měla mít nativní \Cpp{} rozhraní (ačkoliv lze v \Cpp{} relativně jednoduše použít i C knihovny) a být header-only, tedy, že k jejímu použití stačí přidat její hlavičkový soubor a všechny implementační detaily jsou obsaženy v něm.

Nejznámější knihovna využívaná k tomuto účelu je \textit{GNU Readline}\todo{ref?}. \textit{Readline} umožňuje editaci příkazové řádky pomocí mnoha klávesových zkratek, převzatých z textových editorů \textit{EMACS} a \textit{Vi} a také velmi intuitivní inkrementální automatické doplňování (tj.\ postupně doplňování vícenásobným stisknutím klávesové zkratky). Nicméně, vzhledem k tomu, že je již poměrně zastaralá, má pouze rozhraní v jazyce C, tudíž to není dobrá volba pro moji aplikaci. Další nevýhodou je její implementace: \textit{Readline} obsahuje přibližně 20 tisíc řádek kódu především kvůli kompatibilitě s mnoha emulátory terminálů. V dnešní době většina terminálových aplikací podporuje základní VT100 escape sekvence\todo{vysvětlivku} a tudíž není třeba zastaralé terminály podporovat.\todo{ref na linenoise readme.md}

Méně těžkotonážní variantou je knihovna \textit{linenoise}. Ta je oproti \textit{Readline} velmi malá --- má\todo{reword?} zhruba tisíc řádků. Její odnož \textit{cpp-linenoise} napsaná přímo v \Cpp{}, tudíž odpadají různé nevýhody použití jazyka C, a je header-only. Z hlediska funkčních požadavků \textit{cpp-linenoise} bohužel zaostává: nepodporuje mnoho klávesových zkratek (například kombinace klávesy Ctrl a směrových šipek) a automatické doplňování je pouze velmi jednoduché (neinkrementální).  Všechny tyto nedostatky řeší knihovna \textit{replxx}. Ta oproti \textit{linenoise} není header-only, nicméně velmi dobře napodobuje \textit{Readline} z hlediska podpory klávesových zkratek a automatického dokončování. Ve svém programu jsem tedy zvolil knihovnu \textit{replxx}.

% TODO: ukázat docopt.cpp (Ale někde jinde)

\subsection{Zpracování vstupu}
% TODO: intro o vstupu
% TODO: ukázat gramatiky

Ke zpracování gramatik příkazů je potřeba parser\todo{vysvětlivka}. Implementaci parseru lze provést manuálně, to ovšem může být u složitých gramatik velmi nepraktické, a proto je vhodné použít nějakou knihovnu, která dokáže parser vygenerovat. Na základě doporučení \todo{V podstatě jsem ji použil jen tak, bez zvážení, na základě Honzy} jsem použil knihovnu \textit{Boost Spirit X3}. Jednou z hlavních výhod knihovny \textit{Spirit} je, že gramatiky lze zádavat přímo do zdrojového kódu výhradně pomocí prostředků jazyka --- operátorů. To znamená, že není potřeba spouštět žádný preprocesor\todo{vysvětlivka}, který by převedl speciální syntaxi parseru do validního \Cpp{} kódu.
\todo[inline]{Ukázka gramatiky napsané pomocí Spiritu}
% TODO: dopsat gramatiky na základě datového modelu


% TODO: ukázat Boost Spirit
% TODO: ukázat libyang

\subsection{Interpreter}
% TODO: dopsat (vezme user input a zavolá metody/funkce...)

\subsection{DatastoreAccess}
% TODO: dopsat (abstraktní rozhraní pro přístup do datastoru...)

\subsubsection{NetconfAccess}
% TODO: ukázat libnetconf2

\subsubsection{SysrepoAccess}
% TODO: ukázat sysrepo klientskou knihovnu

\subsection{Testování}
% TODO: ukázat doctest
% TODO: ukázat trompeloeil
% TODO: zmínit Catch

\subsection{Knihovna na logování?}
% TODO: spdlog, použito jen v netconf cpp wrapperu... od Honzy


\chapter{Vyhodnocení}
\section{Testování}
% TODO: testování přes doctest, především unit testy, ale i nějaký jiný (sysrepo.cpp)\@. hlášení bugů přes issue tracker, po opravení napsán test
\section{Porovnání s konkurencí}
% TODO: popsat, že ačkoliv moje aplikace nepodporuje 100 \% YANGu/NETCONFu, na základní použití stačí (hlavní cíl byl, upravovat konfiguraci bez znalosti YANG modelů a NETCONFu)
\section{Výsledky z nasazení}
% TODO: no, prej to Honza používá xD

\begin{conclusion}
% TODO: Něco dopsat sem, asi v závislosti na zbytku
\end{conclusion}

\bibliographystyle{csn690}
\bibliography{mybibliographyfile}

\appendix

\chapter{Seznam použitých zkratek}
% \printglossaries
\begin{description}
	\item[GUI] Graphical user interface
	\item[XML] Extensible markup language
\end{description}

\chapter{Obsah přiloženého CD}

%\begin{figure}
%	\dirtree{
%            .1
%	}
%\end{figure}

\end{document}
